\ifx\allfiles\undefined
\documentclass[12pt, a4paper, oneside, UTF8]{ctexbook}
\def\path{../config}
\usepackage{amsmath}
\usepackage{amsthm}
\usepackage{amssymb}
\usepackage{graphicx}
\usepackage{mathrsfs}
\usepackage{enumitem}
\usepackage{geometry}
\usepackage[colorlinks, linkcolor=black]{hyperref}
\usepackage{stackengine}
\usepackage{yhmath}
\usepackage{extarrows}
\usepackage{esint}
\usepackage{fancyhdr}
\usepackage[dvipsnames, svgnames]{xcolor}
\usepackage{listings}

\definecolor{mygreen}{rgb}{0,0.6,0}
\definecolor{mygray}{rgb}{0.5,0.5,0.5}
\definecolor{mymauve}{rgb}{0.58,0,0.82}
\definecolor{NavyBlue}{RGB}{0,0,128}
\definecolor{Rhodamine}{RGB}{255,0,255}
\definecolor{PineGreen}{RGB}{0,128,0}

\graphicspath{ {figures/},{../figures/}, {config/}, {../config/} }

\linespread{1.6}

\geometry{
    top=25.4mm,
    bottom=25.4mm,
    left=20mm,
    right=20mm,
    headheight=2.17cm,
    headsep=4mm,
    footskip=12mm
}

\setenumerate[1]{itemsep=5pt,partopsep=0pt,parsep=\parskip,topsep=5pt}
\setitemize[1]{itemsep=5pt,partopsep=0pt,parsep=\parskip,topsep=5pt}
\setdescription{itemsep=5pt,partopsep=0pt,parsep=\parskip,topsep=5pt}

\lstset{
    language=Mathematica,
    basicstyle=\tt,
    breaklines=true,
    keywordstyle=\bfseries\color{NavyBlue},
    emphstyle=\bfseries\color{Rhodamine},
    commentstyle=\itshape\color{black!50!white},
    stringstyle=\bfseries\color{PineGreen!90!black},
    columns=flexible,
    numbers=left,
    numberstyle=\footnotesize,
    frame=tb,
    breakatwhitespace=false,
}

\lstset{
    language=TeX, % 设置语言为 TeX
    basicstyle=\ttfamily, % 使用等宽字体
    breaklines=true, % 自动换行
    keywordstyle=\bfseries\color{NavyBlue}, % 关键字样式
    emphstyle=\bfseries\color{Rhodamine}, % 强调样式
    commentstyle=\itshape\color{black!50!white}, % 注释样式
    stringstyle=\bfseries\color{PineGreen!90!black}, % 字符串样式
    columns=flexible, % 列的灵活性
    numbers=left, % 行号在左侧
    numberstyle=\footnotesize, % 行号字体大小
    frame=tb, % 顶部和底部边框
    breakatwhitespace=false % 不在空白处断行
}


\usepackage[strict]{changepage}
\usepackage{framed}

\definecolor{greenshade}{rgb}{0.90,1,0.92}
\definecolor{redshade}{rgb}{1.00,0.88,0.88}
\definecolor{brownshade}{rgb}{0.99,0.95,0.9}
\definecolor{lilacshade}{rgb}{0.95,0.93,0.98}
\definecolor{orangeshade}{rgb}{1.00,0.88,0.82}
\definecolor{lightblueshade}{rgb}{0.8,0.92,1}
\definecolor{purple}{rgb}{0.81,0.85,1}

\theoremstyle{definition}
\newtheorem{myDefn}{\indent 定义}[section]
\newtheorem{myLemma}{\indent 引理}[section]
\newtheorem{myThm}[myLemma]{\indent 定理}
\newtheorem{myCorollary}[myLemma]{\indent 推论}
\newtheorem{myCriterion}[myLemma]{\indent 准则}
\newtheorem*{myRemark}{\indent 注}
\newtheorem{myProposition}{\indent 命题}[section]

\newenvironment{formal}[2][]{%
    \def\FrameCommand{%
        \hspace{1pt}%
        {\color{#1}\vrule width 2pt}%
        {\color{#2}\vrule width 4pt}%
        \colorbox{#2}%
    }%
    \MakeFramed{\advance\hsize-\width\FrameRestore}%
    \noindent\hspace{-4.55pt}%
    \begin{adjustwidth}{}{7pt}\vspace{2pt}\vspace{2pt}}{%
        \vspace{2pt}\end{adjustwidth}\endMakeFramed%
}

\newenvironment{definition}{\begin{formal}[Green]{greenshade}\vspace{-\baselineskip / 2}\begin{myDefn}}{\end{myDefn}\end{formal}}
\newenvironment{theorem}{\begin{formal}[LightSkyBlue]{lightblueshade}\vspace{-\baselineskip / 2}\begin{myThm}}{\end{myThm}\end{formal}}
\newenvironment{lemma}{\begin{formal}[Plum]{lilacshade}\vspace{-\baselineskip / 2}\begin{myLemma}}{\end{myLemma}\end{formal}}
\newenvironment{corollary}{\begin{formal}[BurlyWood]{brownshade}\vspace{-\baselineskip / 2}\begin{myCorollary}}{\end{myCorollary}\end{formal}}
\newenvironment{criterion}{\begin{formal}[DarkOrange]{orangeshade}\vspace{-\baselineskip / 2}\begin{myCriterion}}{\end{myCriterion}\end{formal}}
\newenvironment{remark}{\begin{formal}[LightCoral]{redshade}\vspace{-\baselineskip / 2}\begin{myRemark}}{\end{myRemark}\end{formal}}
\newenvironment{proposition}{\begin{formal}[RoyalPurple]{purple}\vspace{-\baselineskip / 2}\begin{myProposition}}{\end{myProposition}\end{formal}}

\newtheorem{example}{\indent \color{SeaGreen}{例}}[section]
\renewcommand{\proofname}{\indent\textbf{\textcolor{TealBlue}{证明}}}
\newenvironment{solution}{\begin{proof}[\indent\textbf{\textcolor{TealBlue}{解}}]}{\end{proof}}


% The custom command, change it according to your needs.

\def\d{\mathrm{d}}
\def\N+{\mathbb{N}^*}
\def\Q{\mathbb{Q}}
\def\Z{\mathbb{Z}}
\def\R{\mathbb{R}}
\newcommand{\myspace}[1]{\par\vspace{#1\baselineskip}}
\newcommand{\xrowht}[2][0]{\addstackgap[.5\dimexpr#2\relax]{\vphantom{#1}}}
\newenvironment{mycases}[1][1]{\linespread{#1} \selectfont \begin{cases}}{\end{cases}}
\newenvironment{myvmatrix}[1][1]{\linespread{#1} \selectfont \begin{vmatrix}}{\end{vmatrix}}
\newcommand{\tabincell}[2]{\begin{tabular}{@{}#1@{}}#2\end{tabular}}
\newcommand{\pll}{\kern 0.56em/\kern -0.8em /\kern 0.56em}
\newcommand{\dive}[1][F]{\mathrm{div}\;\boldsymbol{#1}}
\newcommand{\rotn}[1][A]{\mathrm{rot}\;\boldsymbol{#1}}


\usepackage{tikz}
\usetikzlibrary{positioning}
\definecolor{innerblue}{RGB}{134,159,202}
\definecolor{middleblue}{RGB}{213,222,236}
\definecolor{outerblue}{RGB}{173,190,220}
\definecolor{bggray}{RGB}{215,215,215}
\tikzset{pics/.cd,
	triple circle/.style={code={
			\fill[outerblue] (4,0) circle (4);
			\fill[middleblue] (3.1,0) circle (3.1);
			\fill[innerblue] (2.2,0) circle (2.2);
}}}


\def\myTitle{数学分析讲义(第一册)习题解答}
\def\myAuthor{雨帆}
\def\myDateCover{封面日期:\today}
\def\myDateForeword{前言页显示日期:\today}
\def\myForeword{前言标题}
\def\myForewordText{
    这是一个基于\LaTeX{}的模板,用于撰写学习笔记。

    模板旨在提供一个简单、易用的框架,以便你能够专注于内容,而不是排版细节,如不是专业者,不建议使用者在模板细节上花费太多时间,而是直接使用模板进行笔记撰写。遇到问题,再进行调整解决。
}
\def\mySubheading{副标题}

\begin{document}
\thispagestyle{empty}
\begin{tikzpicture}[overlay,remember picture]
    \fill[bggray] (current page.north west) rectangle (current page.south east);
    \draw[blue!60] ([xshift=-1cm]current page.north)
    -- ++ (-60:9) pic[rotate=90-60,scale=0.7] {triple circle};
    \draw[blue!60] (current page.north west)
    -- ++ (-40:20) pic[rotate=90-40,scale=0.3] {triple circle};
    \draw[blue!60] ([yshift=-11cm]current page.north east)
    -- ++ (-110:15) pic[rotate=90-110,scale=0.8] {triple circle};
    \node[above right=8cm and 2cm of current page.south west,font=\Huge\bfseries,blue!90] (H)
    {\myTitle};
    \node[below=2mm of H.south west,anchor=north west,font=\sffamily]{\myAuthor};
    \node[below=8mm of H.south west,anchor=north west,font=\sffamily]{\myDateCover};
\end{tikzpicture}
\clearpage

\newpage
\thispagestyle{empty}
\begin{center}
    \Huge\textbf{\myForeword}
\end{center}
\myForewordText
\begin{flushright}
    \begin{tabular}{c}
        \myDateForeword
    \end{tabular}
\end{flushright}

\newpage
\pagestyle{plain}
\setcounter{page}{1}
\pagenumbering{Roman}
\tableofcontents

\newpage
\pagenumbering{arabic}
\setcounter{chapter}{-1}
\setcounter{page}{1}

\pagestyle{fancy}
\fancyfoot[C]{\thepage}
\renewcommand{\headrulewidth}{0.4pt}
\renewcommand{\footrulewidth}{0pt}

\fi

\chapter{极限}

\section{实数}

1. 设 $a$ 是有理数,$b$ 是无理数. 求证:$a + b$ 和 $a - b$ 都是无理数;当 $a \neq 0$ 时,$ab$ 和 $b/a$ 也都是无理数.

\begin{solution}
此题可使用反证法进行证明.

(1)假定 $a + b$ 为有理数,则 $\displaystyle a + b = \frac{m}{n}$,其中 $m, n \in \Z$.

因为 $a$ 为有理数,则 $\displaystyle a = \frac{m'}{n'}$,其中 $m', n' \in \Z$.

由此可得 $\displaystyle b = (a + b) - a = \frac{m}{n} - \frac{m'}{n'} = \frac{mn' - m'n}{nn'}$.

这与 $b$ 为无理数的前提相矛盾,所以原假设不成立,$a + b$ 不为有理数.

\vspace{1.5em}

(2)假定 $a - b$ 为有理数,则 $\displaystyle a - b = \frac{m}{n}$,其中 $m, n \in \Z$.

因为 $a$ 为有理数,则 $\displaystyle a = \frac{m'}{n'}$,其中 $m', n' \in \Z$.

由此可得 $\displaystyle b = a - (a - b) = \frac{m'}{n'} - \frac{m}{n} = \frac{m'n - mn'}{nn'}$.

这与 $b$ 为无理数的前提相矛盾,所以原假设不成立,$a - b$ 不为有理数.

\vspace{1.5em}

(3)假定 $ab$ 为有理数,则 $\displaystyle ab = \frac{m}{n}$,其中 $m, n \in \Z$.

因为 $a$ 为有理数且 $a \neq 0$,则 $\displaystyle a = \frac{m'}{n'}$,其中 $m', n' \in \Z$.

由此可得 $\displaystyle b = \frac{ab}{a} = \frac{m}{n}/\frac{m'}{n'} = \frac{mn'}{m'n}$.

这与 $b$ 为无理数的前提相矛盾,所以原假设不成立,$ab$ 不为有理数.

\vspace{1.5em}

(4)假定 $b/a$ 为有理数,则 $\displaystyle b/a = \frac{m}{n}$,其中 $m, n \in \Z$.

因为 $a$ 为有理数且 $a \neq 0$,则 $\displaystyle a = \frac{m'}{n'}$,其中 $m', n' \in \Z$.

由此可得 $\displaystyle b = b/a * a = \frac{m}{n}*\frac{m'}{n'} = \frac{mm'}{nn'}$.

这与 $b$ 为无理数的前提相矛盾,所以原假设不成立,$b/a$ 不为有理数.
\end{solution}

2. 求证:两个不同的有理数之间有有理数.

\begin{solution}
设 $a, b \in \Q$ 满足 $a < b$,则可得不等式 $2a < a + b < 2b$.

令 $\displaystyle c = \frac{a + b}{2}$,则 $a < c < b$.

由此可得,对于任何不相同的有理数 $a, b$,都必然存在有理数 $c$,满足 $a < c < b$.
\end{solution}

3. 求证:$\sqrt{2}$,$\sqrt{3}$ 以及 $\sqrt{2} + \sqrt{3}$ 都是无理数.

\begin{solution}
假定 $\sqrt{2}$ 为有理数,则存在 ${m, n \in \Z}$ 使得 $\displaystyle\sqrt{2} = \frac{m}{n}$,且 $m, n$ 无公约数.

由 $\sqrt{2}$ 的定义可知 $\displaystyle (\sqrt{2})^2 = 2 = \left(\frac{m}{n}\right)^2$. 则可得 $m^2 = 2n^2$,因此 $m^2$ 为偶数.

因为 $m^2$ 为偶数,则 $m$ 必为偶数. 由此可得 $\displaystyle n^2 = \frac{m^2}{2}$ 也必为偶数.

因此 $n$ 也为偶数,其与 $m$ 至少存在一个公约数 $2$,这与原假设矛盾.

所以 $\sqrt{2}$ 不为有理数.

\vspace{1.5em}

假定 $\sqrt{3}$ 为有理数,则存在 ${m, n \in \Z}$ 使得 $\displaystyle\sqrt{3} = \frac{m}{n}$,且 $m, n$ 无公约数.

由 $\sqrt{3}$ 的定义可知 $\displaystyle (\sqrt{3})^2 = 3 = \left(\frac{m}{n}\right)^2$. 则可得 $m^2 = 3n^2$,因此 $m$ 能被 $3$ 整除.

因为 $m$ 能被 $3$ 整除,则 $m^2 = 9z, z \in \mathbb{N}$. 由此可得 $\displaystyle n^2 = \frac{m^2}{3} = 3z$.

因此 $n$ 也能被 $3$ 整除,其与 $m$ 至少存在一个公约数 $3$,这与原假设矛盾.

所以 $\sqrt{3}$ 不为有理数.

\vspace{1.5em}

假定 $\sqrt{2} + \sqrt{3}$ 为有理数,则存在 ${m, n \in \Z}$ 使得 $\displaystyle\sqrt{2} + \sqrt{3} = \frac{m}{n}$,且 $m, n$ 无公约数.

则 $(\sqrt{2} + \sqrt{3})^2 = \left(\frac{m}{n}\right)^2 =  5 + 2\sqrt{6}$

通过反证法易得 $\sqrt{6}$ 为无理数,由问题 1 的结论易得 $5 + 2\sqrt{6}$ 也为无理数.

这与 $\displaystyle \left(\frac{m}{n}\right)^2$ 为有理数的假定相矛盾. 所以 $\sqrt{2} + \sqrt{3}$ 不为有理数.
\end{solution}

4. 把下列循环小数表示为分数:
\[ 0.2499\ 99\cdots,\qquad0.\dot{3}7\dot{5},\qquad4.\dot{5}1\dot{8} \text{.} \]

\begin{solution}
(1)令 $a = 0.2499\ 99\cdots$,则:

\begin{align*}
    1000a & = 249.\dot{9}\\
    100a & = 24.\dot{9}\\
    900a & = 225\\
    a & = \frac{1}{4}
\end{align*}

(2)令 $b = 0.\dot{3}7\dot{5}$,则:

\begin{align*}
    1000b & = 375.\dot{3}7\dot{5}\\
    999b & = 375\\
    b & = \frac{125}{333}
\end{align*}

(3)令 $c = 4.\dot{5}1\dot{8}$,则:

\begin{align*}
    1000c & = 4518.\dot{5}1\dot{8}\\
    999c & = 4514\\
    c & = \frac{4514}{999}
\end{align*}
\end{solution}

5. 设 $r, s, t$ 都是有理数. 求证:\\
\indent(1)若 $r + s\sqrt{2} = 0$,则 $r = s = 0$;\\
\indent(2)若 $r + s\sqrt{2} + t\sqrt{3} = 0$,则 $r = s = t = 0$.

\begin{solution}
(1)因为 $r$ 为有理数,按照定义 $\displaystyle r = \frac{m}{n}$,其中 $m, n \in \Z$. 由此可得:
\[s = \frac{-m}{n\sqrt{2}}\]

由问题 1, 3 的结论可知 $\sqrt{2}$ 为无理数,无理数和任何不为 $0$ 的有理数相乘均为无理数.

则可知 $s$ 为有理数时,$s = 0$,因此 $r = s = 0$.

(2)证明略
\end{solution}

6. 设实数 $a_1, a_2, \cdots, a_n$ 都有相同的符号,且都大于 $-1$. 证明:
\[ (1 + a_1)\ (1 + a_2) \cdots (1 + a_n) \geq 1 + a_1 + a_2 + \cdots + a_n \text{.} \]

\begin{solution}
当 $n = 1$ 时,原不等式为 $(1 + a_1) = 1 + a_1$,成立。

当 $n = 2$ 时,原不等式为 $(1 + a_1)(1 + a_2) = 1 + a_1 + a_2 + a_1a_2 > 1 + a_1 + a_2$,亦成立。

假设当 $n = k, k \in [2, +\infty)$ 时原不等式成立.

因为 $\forall (a_n + 1) > 0$,则可得当 $n = k + 1$ 时存在下列不等式:
\[(1 + a_1)(1 + a_2) \cdots (1 + a_k)(1 + a_{k+1}) \geq (1 + a_1 + a_2 + \cdots + a_k)(1 + a_{k+1})\]

显然可得:
\[(1 + a_1 + a_2 + \cdots + a_k)(1 + a_{k+1}) = (1 + a_1 + a_2 + \cdots + a_k) + (a_{k+1} + a_{1}a_{k+1} + \cdots + a_{k}a_{k+1})\]

因为所有的 $a_n$ 均为相同符号,则可得不等式:
\[(1 + a_1 + a_2 + \cdots + a_k) + (a_{k+1} + a_{1}a_{k+1} + \cdots + a_{k}a_{k+1}) > 1 + a_1 + a_2 + \cdots + a_k + a_{k+1}\]

故 $n = k + 1$ 时,原不等式成立,由此可得原不等式对 $n \in \N+$ 一定成立.
\end{solution}

7. 设 $a, b$ 是实数,且 $|a| < 1, |b| < 1$. 证明:
\[ \left| \frac{a + b}{1 + ab} \right| < 1 \text{.} \]

\begin{solution}
因为 $|a| < 1, |b| < 1$,可知 $-1 < a < 1, -1 < b < 1$.

由此可得 $-1 < ab < 1, 0 < 1 + ab < 2$.

由题设可得:
\begin{gather*}
    \left| \frac{a + b}{1 + ab} \right| < 1\\
    |a + b| < |1 + ab| = 1 + ab\\
    - 1 - ab < a + b < 1 + ab
\end{gather*}

对于 $a + b < 1 + ab$ 可得:
\begin{gather*}
    a + b - 1 - ab < 0\\
    (a - 1)(1 - b) < 0
\end{gather*}

因为 $|a| < 1, |b| < 1$,所以 $a - 1 < 0, 1 - b > 0$,所以 $a + b < 1 + ab$ 成立.

对于 $a + b > -1 - ab$ 可得:
\begin{gather*}
    a + b + 1 + ab > 0\\
    (a + 1)(b + 1) > 0
\end{gather*}

因为 $|a| < 1, |b| < 1$,所以 $a + 1 > 0, b + 1 > 0$,所以 $a + b > -1 - ab$ 成立.

所以题设不等式成立.
\end{solution}

\ifx\allfiles\undefined
\end{document}
\fi
