\ifx\allfiles\undefined
\documentclass[12pt, a4paper, oneside, UTF8]{ctexbook}
\def\path{../config}
\input{\path/package.tex}

\input{\path/theorem.tex}

\input{\path/custom.tex}

\input{\path/cover_package.tex}

\def\myTitle{数学分析讲义(第一册)习题解答}
\def\myAuthor{雨帆}
\def\myDateCover{日期:\today}
\def\myDateForeword{}
\def\myForeword{简介}
\def\myForewordText{
    本文档用于尝试对程艺老师编写的《数学分析讲义(第一册)》的习题进行解答,属于作者的学习笔记。本文档的答案不具备任何可参考性。
}
\def\mySubheading{}

\begin{document}
\input{../config/cover}
\fi

\chapter{极限}

\section{实数}

1. 设 $a$ 是有理数,$b$ 是无理数. 求证:$a + b$ 和 $a - b$ 都是无理数;当 $a \neq 0$ 时,$ab$ 和 $b/a$ 也都是无理数.

\begin{solution}
此题可使用反证法进行证明.

(1)假定 $a + b$ 为有理数,则 $\displaystyle a + b = \frac{m}{n}$,其中 $m, n \in \mathbb{N}$.

因为 $a$ 为有理数,则 $\displaystyle a = \frac{m'}{n'}$,其中 $m', n' \in \mathbb{N}$.

由此可得 $\displaystyle b = (a + b) - a = \frac{m}{n} - \frac{m'}{n'} = \frac{mn' - m'n}{nn'}$.

这与 $b$ 为无理数的前提相矛盾,所以原假设不成立,$a + b$ 不为有理数.

\vspace{1.5em}

(2)假定 $a - b$ 为有理数,则 $\displaystyle a - b = \frac{m}{n}$,其中 $m, n \in \mathbb{N}$.

因为 $a$ 为有理数,则 $\displaystyle a = \frac{m'}{n'}$,其中 $m', n' \in \mathbb{N}$.

由此可得 $\displaystyle b = a - (a - b) = \frac{m'}{n'} - \frac{m}{n} = \frac{m'n - mn'}{nn'}$.

这与 $b$ 为无理数的前提相矛盾,所以原假设不成立,$a - b$ 不为有理数.

\vspace{1.5em}

(3)假定 $ab$ 为有理数,则 $\displaystyle ab = \frac{m}{n}$,其中 $m, n \in \mathbb{N}$.

因为 $a$ 为有理数且 $a \neq 0$,则 $\displaystyle a = \frac{m'}{n'}$,其中 $m', n' \in \mathbb{N}$.

由此可得 $\displaystyle b = \frac{ab}{a} = \frac{m}{n}/\frac{m'}{n'} = \frac{mn'}{m'n}$.

这与 $b$ 为无理数的前提相矛盾,所以原假设不成立,$ab$ 不为有理数.

\vspace{1.5em}

(4)假定 $b/a$ 为有理数,则 $\displaystyle b/a = \frac{m}{n}$,其中 $m, n \in \mathbb{N}$.

因为 $a$ 为有理数且 $a \neq 0$,则 $\displaystyle a = \frac{m'}{n'}$,其中 $m', n' \in \mathbb{N}$.

由此可得 $\displaystyle b = b/a * a = \frac{m}{n}*\frac{m'}{n'} = \frac{mm'}{nn'}$.

这与 $b$ 为无理数的前提相矛盾,所以原假设不成立,$b/a$ 不为有理数.
\end{solution}

2. 求证:两个不同的有理数之间有有理数.

\begin{solution}
设 $a, b \in \R$ 满足 $a < b$,则可得不等式 $2a < a + b < 2b$.

令 $\displaystyle c = \frac{a + b}{2}$,则 $a < c < b$.

由此可得,对于任何不相同的有理数 $a, b$,都必然存在有理数 $c$,满足 $a < c < b$.
\end{solution}

3. 求证:$\sqrt{2}$,$\sqrt{3}$ 以及 $\sqrt{2} + \sqrt{3}$ 都是无理数.

\begin{solution}

\end{solution}

4. 把下列循环小数表示为分数:
\[ 0.2499\ 99\cdots,\qquad0.\dot{3}7\dot{5},\qquad4.\dot{5}1\dot{8} \text{.} \]

\begin{solution}
(1)令 $a = 0.2499\ 99\cdots$,则:

\begin{align*}
    1000a & = 249.\dot{9}\\
    100a & = 24.\dot{9}\\
    900a & = 225\\
    a & = \frac{1}{4}
\end{align*}

(2)令 $b = 0.\dot{3}7\dot{5}$,则:

\begin{align*}
    1000b & = 375.\dot{3}7\dot{5}\\
    999b & = 375\\
    b & = \frac{125}{333}
\end{align*}

(3)令 $c = 4.\dot{5}1\dot{8}$,则:

\begin{align*}
    1000c & = 4518.\dot{5}1\dot{8}\\
    999c & = 4514\\
    c & = \frac{4514}{999}
\end{align*}
\end{solution}

5. 设 $r, s, t$ 都是有理数. 求证:\\
\indent(1)若 $r + s\sqrt{2} = 0$,则 $r = s = 0$;\\
\indent(2)若 $r + s\sqrt{2} + t\sqrt{3} = 0$,则 $r = s = t = 0$.

\begin{solution}

\end{solution}

6. 设实数 $a_1, a_2, \cdots, a_n$ 都有相同的符号,且都大于 $-1$. 证明:
\[ (1 + a_1)\ (1 + a_2) \cdots (1 + a_n) \geq 1 + a_1 + a_2 + \cdots + a_n \text{.} \]

\begin{solution}

\end{solution}

7. 设 $a, b$ 是实数,且 $|a| < 1, |b| < 1$. 证明:
\[ \left| \frac{a + b}{1 + ab} \right| < 1 \text{.} \]

\begin{solution}

\end{solution}

\ifx\allfiles\undefined
\end{document}
\fi
