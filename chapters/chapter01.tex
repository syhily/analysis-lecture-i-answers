\ifx\allfiles\undefined
\documentclass[12pt, a4paper, oneside, UTF8]{ctexbook}
\def\path{../config}
\input{\path/package.tex}

\input{\path/theorem.tex}

\input{\path/custom.tex}

\input{\path/cover_package.tex}

\def\myTitle{数学分析讲义(第一册)习题解答}
\def\myAuthor{雨帆}
\def\myDateCover{日期:\today}
\def\myDateForeword{}
\def\myForeword{简介}
\def\myForewordText{
    本文档用于尝试对程艺老师编写的《数学分析讲义(第一册)》的习题进行解答,属于作者的学习笔记。本文档的答案不具备任何可参考性。
}
\def\mySubheading{}

\begin{document}
\input{../config/cover}
\fi

\chapter{极限}

\section{实数}

1. 设 $a$ 是有理数,$b$ 是无理数. 求证:$a + b$ 和 $a - b$ 都是无理数;当 $a \neq 0$ 时,$ab$ 和 $b/a$ 也都是无理数.

\begin{solution}

\end{solution}

2. 求证:两个不同的有理数之间有有理数.

3. 求证:$\sqrt{2}$,$\sqrt{3}$ 以及 $\sqrt{2} + \sqrt{3}$ 都是无理数.

4. 把下列循环小数表示为分数:
\[ 0.2499\ 99\cdots\text{,}\;0.\dot{3}7\dot{5}\text{,}\;4.\dot{5}1\dot{8} . \]

\begin{solution}
令 $a = 0.2499\ 99\cdots$
\end{solution}

5. 设 $r\text{,}s\text{,}t$ 都是有理数. 求证:\\
\indent(1)若 $r + s\sqrt{2} = 0$,则 $r = s = 0$;\\
\indent(2)若 $r + s\sqrt{2} + t\sqrt{3} = 0$,则 $r = s = t = 0$.

6. 设实数 $a_1\text{,}a_2\text{,}\cdots\text{,}a_n$ 都有相同的符号,且都大于 $-1$. 证明:
\[ (1 + a_1)\ (1 + a_2) \cdots (1 + a_n) \geq 1 + a_1 + a_2 + \cdots + a_n . \]

7. 设 $a\text{,}b$ 是实数,且 $|a| < 1\text{,}|b| < 1$. 证明:
\[ \left| \frac{a + b}{1 + ab} \right| < 1 . \]

\ifx\allfiles\undefined
\end{document}
\fi
